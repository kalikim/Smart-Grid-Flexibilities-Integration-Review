%%
%% This is file `sample-authordraft.tex',
%% generated with the docstrip utility.
%%
%% The original source files were:
%%
%% samples.dtx  (with options: `authordraft')
%% 
%% IMPORTANT NOTICE:
%% 
%% For the copyright see the source file.
%% 
%% Any modified versions of this file must be renamed
%% with new filenames distinct from sample-authordraft.tex.
%% 
%% For distribution of the original source see the terms
%% for copying and modification in the file samples.dtx.
%% 
%% This generated file may be distributed as long as the
%% original source files, as listed above, are part of the
%% same distribution. (The sources need not necessarily be
%% in the same archive or directory.)
%%
%% The first command in your LaTeX source must be the \documentclass command.
\documentclass[nonacm,sigconf,12pt]{acmart}
\settopmatter{printacmref=false} % Removes citation information below abstract
\pagestyle{plain} % removes running headers
\renewcommand\footnotetextcopyrightpermission[1]{} % removes footnote with conference information in first column
\usepackage{natbib,hyperref}

\usepackage{graphicx}
\graphicspath{ {./flex_images/} }
\pagestyle{plain} % removes running headers
%% NOTE that a single column version may be required for 
%% submission and peer review. This can be done by changing
%% the \doucmentclass[...]{acmart} in this template to 
%% \documentclass[manuscript,screen,review]{acmart}
%% 
%% To ensure 100% compatibility, please check the white list of
%% approved LaTeX packages to be used with the Master Article Template at
%% https://www.acm.org/publications/taps/whitelist-of-latex-packages 
%% before creating your document. The white list page provides 
%% information on how to submit additional LaTeX packages for 
%% review and adoption.
%% Fonts used in the template cannot be substituted; margin 
%% adjustments are not allowed.
%%
%% \BibTeX command to typeset BibTeX logo in the docs

% Define a new command for ORCID
% Define a new command for ORCID
\newcommand{\myorcid}[1]{\href{https://orcid.org/#1}{\texttt{ORCID: #1}}}



%%
%% Submission ID.
%% Use this when submitting an article to a sponsored event. You'll
%% receive a unique submission ID from the organizers
%% of the event, and this ID should be used as the parameter to this command.
%%\acmSubmissionID{123-A56-BU3}

%%
%% The majority of ACM publications use numbered citations and
%% references.  The command \citestyle{authoryear} switches to the
%% "author year" style.
%%
%% If you are preparing content for an event
%% sponsored by ACM SIGGRAPH, you must use the "author year" style of
%% citations and references.
%% Uncommenting
%% the next command will enable that style.
%%\citestyle{acmauthoryear}

%%
%% end of the preamble, start of the body of the document source.
\setcopyright{none}
\begin{document}
\settopmatter{printacmref=false}
%%
%% The "title" command has an optional parameter,
%% allowing the author to define a "short title" to be used in page headers.
\title{FLEXIBILITIES REVIEW: INTEGRATION INTO SMART GRIDS }

%%
%% The "author" command and its associated commands are used to define
%% the authors and their affiliations.
%% Of note is the shared affiliation of the first two authors, and the
%% "authornote" and "authornotemark" commands
%% used to denote shared contribution to the research.

\author{Anthony Kimanzi\\\myorcid{0000-0001-2345-6789}}



%%
%% By default, the full list of authors will be used in the page
%% headers. Often, this list is too long, and will overlap
%% other information printed in the page headers. This command allows
%% the author to define a more concise list
%% of authors' names for this purpose.




%%
%% This command processes the author and affiliation and title
%% information and builds the first part of the formatted document.
\begin{abstract}
    In the recent past, there has been a huge pressure on the power system as there is an increased demand for energy consumption. Power systems have not only been faced with supply and demand pressure but also been faced with the problem of ensuring the supply of clean and reliable energy. need for flexibility in the system is required with the introduction of renewable energies. Integration of renewable energy into existing architecture was not easy but with the birth of smart grids, there have been endless opportunities and optimization of power systems this paper focuses on virtual power plants(VPP). It starts with a brief overview of smartgrids, components of smartgrids, analysis into how virtual VPP works, integration of VPP into Smart Grid Architecture Model (SGAM) architecture and current VPP projects.
\end{abstract}
\maketitle

\section*{Introduction}
The energy sector is in the middle of the transition, the power generation is shifting from centralized fossil-fueled generation to decentralized renewable generation at the same time the road transport is moving from internal combustion engine towards battery electric vehicles. While these technologies are as good as they lower emissions and lower costs, they can cause some significant side effects. 

From the supply side, we are moving from stable and controllable loads towards more volatile and less predictable power generation sources. On top of that, the volatility has to be mitigated by the decreasing traditional power generators on the demand side. The charging of electric vehicles puts a lot of pressure on the grid because of their high demand correlating charging, with over one million public charging plugs for electric-vehicles as indicated by BloombergNEF as of August 2020 \cite{stock_2020_global}. The combination of these developments makes it harder and harder for the system operators and utilities to balance out supply and demand to ensure a stable and reliable grid. 

Demand-side response innovations can help to shave peaks, fill valleys, and shift loads, \cite{arteconi_2018_assessing} this will allow non-traditional stakeholders such as large commercial and industrial energy consumers to capture the value of flexibilities. Battery storage can serve on a residential scale as part of the demand side and on a utility-scale to help to stabilize supply. Excessive power can be converted into a variety of forms gas, heat, hydrogen, or can be used to pump hydro and to be re-converted at a later moment in time when it is needed most.

Advanced power plant flexibility and virtual power plants help to enhance the adaptability of the power supply. Charging electric vehicles at times of low demand or high supply, or even using their batteries to charge back into the grid can enable electric-vehicles to be an effective resource to the grid. 

Supply and demand can differ over geographical areas, so coupling of markets and networks can help to balance the grid. Emerging technologies such as blockchain, artificial intelligence, and machine learning will provide tools to improve current or even enable completely new solutions. This has all been enabled due to the smart-grid. This paper starts with a short overview of what is smart-grids and flexibilities then narrows down to virtual power plants.

\section*{background}
\subsection*{Smartgrids and Virtual Power Plants (VPPs)}
The smart-grid Technologies are creating new methods and opportunities to support the energy distribution system and enhance reliability by establishing precise control and planning through state estimation.\cite{huang_2012_state} 

Smart grid technologies can support new energy distribution systems by the provision of additional flexibility from the aggregated distributed energy resources(DER) by design to provide Frequency Control Ancillary Services (FCAS) that ensure stability and security of supply\cite{riesz_2015_frequency}. And this satisfies the worrying question about sufficient operational flexibility at the transmission level\cite{mayorgagonzalez_2018_determination}.

Advances in communication, computing, and sensor technology are creating new alternative solutions to traditional approaches to meeting the ever-increasing demand. Virtual Power plants improve grid reliability as an alternative to install additional infrastructure.

\begin{figure}
 \textbf{VPP Illustration}\par\medskip
 \includegraphics[scale=0.5,width=8cm]{flex_images/vpp-illustration.jpg}
 \caption{A large variety of resources incorporated into a virtual power plant (VPP). The interconnected units can then be dispatched using special software and traded intelligently on the energy market. Courtesy: ABB \cite{power_2020_the}}
 \label{fig:vppIllustration}
\end{figure}
In Figure \ref{fig:vppIllustration} It illustrates a variety of resources(Wind farm,Power grid, Solar power plants , Storage, Biomass powerplant) aggregated into a VPP and the interconnected units that can be dispatched through software and intelligently traded on energy market.\cite{power_2020_the}
\subsection*{Components of VPP}
An ideal VPP consist of three components Generation technology, Energy Storage technologies and information communication Technology(ICT) \cite{saboori_2011_virtual}. \newline 
\textbf{Generation Technology} \newline
These consist consist of mainly power generation units under decentralized generation category which generate electricity for use on site \cite{eesienvironmentalandenergystudyinstitute_distributed} . The DER considered for VPP are mainly Wind, solar, Combined Heat and Power (CHP), waste to energy ( Biomass and Biogas), small hydro power plants 
\textbf{Energy storage Technologies} \newline
For the operation of power systems  the energy storage systems (ESS) are essential as they ensure continuity of energy flow and improve  the reliability of the power system. ESS have been an enabler for high penetration of variable renewable generation like solar and wind \cite{jafari_2020_power}. ESS integrated in VPP include :\cite{gharehpetian_2017_distributed} \cite{saboori_2011_virtual}
\begin{enumerate}
    \item Battery ESS (BESS)
    \item Compressed air energy storage(CAES)
    \item Flywheel ESS(FESS)
    \item Pumped hydroelectric
    \item Superconducting magnetic energy storage (SMES)
    \item Ultracapacitor
\end{enumerate}

\textbf{Information communication Technology (ICT)} \newline
ICT systems and  infrastructures are key component to VPP. Explicitily provided by the Smartgrids \cite{potencianomenci_2020_a}. In VPP media technologies are considered for “communications in Energy Management Systems (EMS), Supervisory Control and Data Acquisition (SCADA) \cite{energysystems_scada} and Distribution Dispatching Center (DCC)” \cite{saboori_2011_virtual}, \cite{vilcahuamn_2020_interactive}

\section*{Analysis}
\subsection*{What are virtual Power Plants and How they Operate?}
According to next-Kraftwerk, “A VPP is a network of decentralized, medium-scale power generating units such as wind farms, solar parks, and Combined Heat and Power (CHP) units, as well as flexible power consumers and storage systems.”\cite{nextkraftwerke_2019_} By consolidating communications infrastructure, intelligence, and sensors big selection of distributed energy resources(DERs) may be consolidated and treated as VPP. This involves the collection of knowledge from over many DERs via a secure communication infrastructure. These results are aggregated and controlled in a sort of a traditional power station although they continue to be independent of operation and ownership.

Due to the bidirectional flow of communication between individual power plants, this not only facilitates for transmission of control commands but also real-time data delivery to the control system and this can be used for precise electricity trading forecast and scheduling of flexible power plants.

VPP is connected to one central system, they consist mainly of DERs like hydropower, solar power, biomass Wind energy coupled with demand-side management and storage to make accurate forecasting, optimization, and dispatching of their generation and consumption. \cite{nextkraftwerke_2017_vpp}

The VPP provides balance to the power fluctuations arising variables renewables generations like solar and wind by ramping up and down power generations and power consumptions of controllable units.

virtual power plants have made it possible to integrate renewable sources into the existing energy markets. by integrating many renewables the virtual power plant can trade as a single entity and it can meet the market demands of availability and reliability.

\subsection*{Integration of VPP into the Smart Grid Architecture Model  (SGAM).}
\subsection*{VPP Business overview}  
To integrate the VPP in the Smart Grid Architecture Model (SGAM) this paper considers a remote-controlled VPP \cite{integratingthesmartgridsaustria_2017_technical}. Four different entities are involved: “the VPP operator  (VPPOP), the distributed energy units (DEUs) constituting the VPP via their operation and control units (DEUOP, DEUC), the energy market, and the distribution system operators (DSOs) to whom the DEUs are electrically connected” \cite{integratingthesmartgridsaustria_2017_technical} In figure \ref{fig:vppOverview} a detailed 3 dimension VPP business overview is illustrated on the SGAM:  \newline
\textbf{Virtual Power Plant Operator (VPPOP)} \newline
This represents the central control center. It creates aggregated forecasts to trade energy on the energy market. In the SGAM it is located in the DER domain and enterprise and operation zone \cite{integratingthesmartgridsaustria_2017_technical} \newline
\textbf{Distributed Energy Unit Operator (DEUOP)} \newline
It controls a local group of DEUs. in the SGAM it is located in the station zone.\newline
\textbf{Distributed Energy Unit Controller (DEUC)} \newline
It represents an addressable control interface that controls a specific DEU. In the SGAM it is located in the field zone.\newline
\textbf{Distributed Energy Unit (DEU)} \newline
It produces, consumes, or stores energy. in the SGAM it is located in the DER domain and process zone. \newline
\textbf{Distribution System Operator (DSO)} \newline
The DSO owns and manages the electric power grid distribution. The DSO is also responsible for safe and reliable operations of electric power in the distribution grid. \newline
\textbf{Energy Exchange} \newline
Represents the energy marketplace for buying and selling of the electric energy.
\begin{figure}
 \textbf{VPP Business Overview}\par\medskip
 \includegraphics[scale=0.5,width=8cm]{flex_images/vpp-sgam.jpg}
 \caption{3Dimension illustration of VPP business Overview on the SGAM}
 \label{fig:vppOverview}
\end{figure}
\subsection*{VPP Business Use Cases} 
From the business overview eleven use cases can be defined according to IEC 62559\cite{integratingthesmartgridsaustria_2017_technical} \cite{systems_use} The figure \ref{fig:vppUsescases} shows an overview of the VPP business Use Cases on the SGAM:
\begin{itemize}
    \item VPP00-EstablishtheVPP
    \item VPP01- Create individual forecast for the VPP
    \item VPP02- Exchange the VPP forecast with the DSO
    \item VPP03- Participate on the Energy Exchange
    \item VPP04- Send schedule to the DEUOP
    \item VPP05- Send planned schedule to the DSO
    \item VPP06- Send actual schedule to the VPPOP
    \item VPP07- Transmit adjustments on the planned schedule to the VPPOP
    \item VPP08- Renegotiate schedule on the Energy Exchange
    \item VPP09- Provide measured values and meter data by the DEUC
    \item VPP10- Fix imbalances in the electric power system directly by the DSO
\end{itemize}

\begin{figure}
 \textbf{VPP Business Use Cases}\par\medskip
 \includegraphics[scale=0.5,width=8cm]{flex_images/vpp-usecases.png}
 \caption{Overview of VPP business cases on the SGAM}
 \label{fig:vppUsescases}
\end{figure}


\subsection*{Benefits of VPPs}
The VPP have provide wide benefits to stakeholders, The stakeholders have been grouped into the following groups:-
\begin{enumerate}
    \item Owners of DER units
    \item suppliers and aggregators
    \item Transmission System Operators (TSOs) and Distribution System Operators (DSOs)
    \item Policy Makers
    
\end{enumerate}
Key benefit to each stakeholder have been identified as follows \cite{Braun2009} \newline
\textbf{Main benefits for owners of DER units:}
\begin{itemize}
    \item Utilize the benefits of flexibility 
    \item Increasing value of assets through the markets
    \item Aggregation of the DER units reduces the financial risk to individual  DER unit
    \item Raise chances for the DER unit owners to negotiate energy market conditions 
    
\end{itemize}

\textbf{Main benefits for DSOs and TSOs: }
\begin{itemize}
    \item Ability to utilize control flexibility of DER units for network management
    \item improved visibility of DER units for consideration in network operation 
    \item Improved use of grid investments 
    \item Improved co-ordination between DSO and TSO
    \item Mitigate the complexity of operation caused by the growth of inflexible distributed generation 
\end{itemize}

\textbf{Main benefits for Policy Makers:} 
\begin{itemize}
    \item Cost-effective  scalability  and integration of large-scale renewable energies while maintaining system security 
    \item Open the energy markets to small-scale participants
    \item Increasing the increasing worldwide  efficiency of the electrical power system by utilizing the  flexibility of DER units
    \item enabler for penetrations and reaching set targets for the deployment of renewable energies with reduced CO2 emissions.
    \item Improve consumer choice 
    \item it is a source of employment 
    
\end{itemize}

\textbf{Main benefits for suppliers and aggregators :} 
\begin{itemize}
    \item New offers for consumers and DER units. 
    \item Mitigating commercial risk.
    \item It is a source of  business opportunities.
\end{itemize}

\subsection*{Examples of VPPs}
\begin{enumerate}
  \item \textbf{Next Kraftwerke} \newline
  it is a virtual power plant operator based in Cologne Germany with a networked capacity of 8,538 MegaWatts(MW) and comprises 9,966 aggregated units as of September 2020. Next Kraftwerke was founded in the year 2009. \cite{nextkraftwerke_virtual}
  \item \textbf{Statkraft} \newline
  It’s considered Europe's biggest virtual power plant with a capacity of 10,000MW and comprising of 1400 independent power producers. It began in Norway and its global headquarters are in Oslo, Norway. \cite{actionrenewablescouk_2020_virtual}
  \item \textbf{Centrica and Sonnen} \newline
  Centrica and Sonnen operate 2.5 Gigawatts(GW) virtual power plants in Europe, Asia, and North America as of September 2019. \cite{centrica_stackpath}
  \item \textbf{Simply energy Virtual power plant Adelaide} \newline
  It is a VPP based in Adelaide South Australia, it has a capacity of 8MW\cite{theguardian_2016_adelaide}
\end{enumerate}


\section*{Conclusion}
Virtual power plants(VPP’s) are quite beneficial to the stakeholders, with the continuing advancement of Technology and a   future centered on sustainable renewable energies, significant growth of VPP across the globe is expected.  This growth will be driven by the need for economic growth as VPPs are avenues for new business opportunities and the need for curbing global warming as VPPs help in integrating renewables energies contributing to reduced carbon dioxide (CO2) emissions.

\bibliographystyle{unsrtnat}
\bibliography{flexibility}

\end{document}
\endinput


